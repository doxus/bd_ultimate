  % !TeX spellcheck = pl_PL
  \documentclass[a4paper,twoside]{article}
  \usepackage{polski}
  \usepackage[utf8]{inputenc}
  \usepackage{graphicx}
  \usepackage{amsmath}

  \usepackage[unicode, bookmarks=true]{hyperref} %do zakładek

  \setlength{\textheight}{24cm}
  \setlength{\textwidth}{15.92cm}
  \setlength{\footskip}{10mm}
  \setlength{\oddsidemargin}{0mm}
  \setlength{\evensidemargin}{0mm}
  \setlength{\topmargin}{0mm}
  \setlength{\headsep}{5mm}

  \setcounter{secnumdepth}{3}	%max numerowany poziom zagniezdzenia
  \setcounter{tocdepth}{3}	%uwzglednienie w spisie tresci

  \begin{document}
  \bibliographystyle{plain}

  \begin{titlepage}
  \title{\huge Bazy Danych II - ULTIMATE}
  \author{\large SonMati \\ \large Doxus}
  \maketitle
  \end{titlepage}

  %===============================================================================
  %*** PYTANIA I ODPOWIEDZI ******************************************************
  %===============================================================================
  \part*{Pytania i odpowiedzi}
  % pytanie 1
  \section{Co to jest ROWID?}
      \begin{itemize}
      \item[] \textbf{umożliwia identyfikację rekordu} \emph{tak, bo określa jednoznacznie wiersz w tabeli (a nawet jego wersję)}
      \item[] \textbf{umożliwia wersjonowanie (wykrywanie modyfikacji)} \emph{tak, bo zmiany rekordu są w nim odnotowywane}
      \item jest przeznaczony do tworzenia kluczy głównych
      \item \textbf{jest w każdej tabeli} \emph{tak, bo jest po prostu w każdej tabeli xD}
      \end{itemize}

  \section{W SQL Base realizacja fazy}
      \begin{itemize}
      \item \textbf{prepare - wysyłanie tekstu komendy do servera}
      \item \textbf{execute - wysłanie zmiennych do servera}
      \item execute - wypracowanie planu wykonania \emph{??to sobie sam serwer robi, a nie faza zapytania}
      \item \textbf{prepare - sprawdzenie uprawnień usera do obiektów}
      \end{itemize}

  \section{W modelu logicznym}
      \begin{itemize}
      \item są definicje indeksów
      \item nie może być związków n..m
      \item występują klucze obce
      \item nie mogą wystąpić związki 1..1
      \end{itemize}

  \section{SQL base na pozimie RL}
      \begin{itemize}
      \item blokada w trybie dzielonym wszystkich wierszy odczytanych w bieżącej transakcji \emph{RL nie posiada trwałego S-LOCKa - blokady w trybie dzielonym.}
      \item blokada w trybie wyłącznym wierszy zmodyfikowanych (chyba też do końca transakcji) \emph{?? RL nie zakłada X-LOCKA - blokady w trybie wyłącznym.}
      \item \textbf{blokada w trybie dzielonym tylko jednego (ostatnio odczytanego) wiersza do czasu odczytania kolejnych danych} \emph{RL polega na krótkim nałożeniu S-LOCKa na wiersz, odczytaniu danych i zdjęciu go zaraz po odczytcie.}
      \item umożliwia odczyt danych niezależnie od blokad nałożonych w innych transakcjach \emph{?? RL jest zależne od S-LOCKa.}
      \end{itemize}

  % pytanie 5    
  \section{Kiedy serwer bazy danych zdejmuje blokady}
      \begin{itemize}
      \item \textbf{po COMMIT} \emph{generalnie trzeba zakończyć transakcję}
      \item gdy wykonujemy następny CONNECT z aktualnie używanej aplikacji
      \item w fazie prepare gdy następuje ona po instrukcjach DML
      \item \textbf{po ROLLBACK} \emph{trzeba zakończyć transakcję}
      \end{itemize}

  \section{SQL base na pozimie RL}
      \begin{itemize}
      \item blokada w trybie dzielonym wszystkich wierszy odczytanych w bieżącej transakcji
      \item \textbf{blokada w trybie dzielonym zawsze tylko jednego (ostatnio odczytanego) wiersza do czasu odczytania kolejnych danych}
      \end{itemize}

  \section{Co to jest ROWID: służy m.in.}
      \begin{itemize}
      \item do ułatwienia wyszukiwania rekordów
      \item \textbf{do wykrywania zmian w rekordach} \emph{bo przy zmianie rekordu zastępuje zmiana ROWID, więc różne ROWID tego samego rekordu oznaczają zmianę w jego treści}
      \item  do nadawania indeksów 
      \item \textbf{czy jest w każdej tabeli.} \emph{jest w każdej tabeli i tyle}
      \end{itemize}

  \section{Model logiczny - co występuje, a co nie}
      \begin{itemize}
      \item \textbf{relacje 1..1}
      \item \textbf{relacje n..n}
      \item indeksy
      \item klucze obce cycki
      \end{itemize}

  \section{Kiedy system zdejmuje blokady}
      \begin{itemize}
      \item \textbf{commit} \emph{zakończenie transakcji}
      \item \textbf{rollback} \emph{zakończenie transakcji}
      \item \textbf{faza prepare następnego zapytania}
      \item \textbf{nowy connect na tym samym uchwycie} \emph{?? reconnect?}
      \end{itemize}

  % pytanie 10    
  \section{W implementacji mechanizmu transakcyjnego w systemie DBMS SQLBase}
      \begin{itemize}
      \item \textbf{zwolnienie wszelkich nałożonych blokad w bieżącej transakcji, następuje w sytuacji wykonania polecenia ROLLBACK w bieżącej transakcji}
      \end{itemize}

  \section{W systemie transakcyjnym DBMS SQLBase}
      \begin{itemize}
      \item system podziału transakcji (mechanizm SAVEPOINT- ROLLBACK ) pozwala na wycofywanie i zatwierdzanie fragmentów transakcji
      \end{itemize}

  \section{Kolumna ROWID w systemie zarządzania bazą danych SQLBase}
      \begin{itemize}
      \item \textbf{jest kolumną dostępną w każdej tabeli}
      \end{itemize}

  \section{Określ prawdziwość zdań dotyczących realizacja fazy zapytania/zdania SQL w systemie zarządzania bazą danych SQLBase}
      \begin{itemize}
      \item \textbf{faza Execute powoduje przesłanie wartości zmiennych wiążących – wartości parametrów komendy SQL}
      \end{itemize}


  %===============================================================================
  %*** MATERIAŁY WYKŁADOWE *******************************************************
  %===============================================================================
  \part*{Podstawy}
  \section*{Podstawowe pojęcia}
      \begin{itemize}
      \item \textbf{Strona} - Data Row Page.
      \item \textbf{Tablica} - dwukierunkowa lista stron.
      \item \textbf{Slot} - miejsce, gdzie może leżeć wiersz.
          \begin{itemize}
          \item Mogą to być różne pozycje, ponieważ długość wiersza nie jest zdefiniowana.
          \item Jeżeli strona nie jest w pełni wypełniona, zostają puste miejsca.
          \item Są numerowane, każdy posida Update Number.
          \item Każdy posiada wskaźnik do extentu (osobny typ strony).
          \end{itemize}
      \item \textbf{Header} - zawiera informacje o slotach.
      \item \textbf{Footer} - suma kontrolna (CRC, bit parzystości, klucz, szyfrowanie). \emph do uzupełnienia
      \end{itemize}

  % --- kolumny ROWID
  \part*{Kolumny ROWID}
  \section*{Co to jest?}
      Unikalna identyfikacja każdego wiersza, wartości w kolumnie ROWID jednoznacznie identyfikują wiersz w tablicy.ROWID definiujemy jako

  \section*{Właściwości}
      \begin{itemize}
      \item Każda tablica posiada domyślnie kolumnę ROWID (obsługiwaną przez serwer bazy danych), którą można odczytywać w ramach wykonywania komend SELECT, UPDATE, DELETE
      \item Po zmianie jakiejkolwiek wartości w wierszu następuje zmiana wartości ROWID (z zachowaniem unikalności w ramach tablicy) – versioning
      \item Budowa ROWID: identyfikator powstaje na skutek sklejenia odpowiednich wartości: page number, slot number, row serial number (insert sn), update serial number
        \begin{center}
        \(ROWID = PageNumber  + InsertNumber + SlotNumber + UpdateNumber\)
        \end{center}
      \item Mimo to jest typu STRING i można go używać w zdaniach SQL tak jak nazwy kolumny
      \item Zmienia wartość po każdej zmianie wiersza - należy korzystać z nich ostrożnie, zwłaszcza w formularzach \emph{master-detail}
      \end{itemize}
  \section*{Wykorzystanie}
      \begin{itemize}
      \item Ponieważ ROWID jednoznacznie identyfikuje wiersz, wykorzystuje się go przy tworzeniu wynikowej tablicy dla zapytań (Result Set)
      \item W wypadku, kiedy zapytanie wymaga złączenia kilku tablic
      \item Do koordynacji działania dwóch kursorów
      \end{itemize}

  % ======================================================================
  % === POZIOMY IZOLACJI
  % ======================================================================
  \part*{Poziomy izolacji oraz blokady}


  \section*{Co to jest?}
      Mechanizm, który umożliwia wielodostęp poprzez zastosowanie izolacji oraz blokad.

  % --- właściwości izolacji
  \section*{Właściwości}
      \begin{itemize}
      \item Ustawienie poziomu izolacji decyduje o sposobie izolowania i blokowania danych 
      \item Dla aplikacji istotnie wielodostępnych poziomy izolacji muszą być dobrane do okoliczności 
      \item Blokowanie zapisu następuje na poziomie stron 
      \item Blokady są zwalniane po COMMIT / ROLLBACK 
      \item Poziomy izolacji ustala się:
      \begin{itemize}
          \item Poleceniem \emph{SET ISOLATION 'poziom'} w SQLTalk 
          \item Funkcją \emph{SqlSetIsolationLevel} w SQLWindows
      \end{itemize}
      \item Każdy kursor otwarty przez klienta ma własny bufor wejściowy
      \end{itemize}

  % --- blokady
  \section*{Blokady}				
      \begin{itemize}
      \item \textbf{S-LOCK} - blokada w trybie dzielonym (shared), współdzielenie odczytu – każda transakcja może zablokować wybrany wiersz (rekord) i go odczytywać, Jeżeli chce go zmodyfikować, musi podnieść poziom blokady na X-LOCK, ale wtedy musi poczekać, aż inne transakcje zdejmą swoje S-LOCKi.
      \item \textbf{X-LOCK} - blokada w trybie wyłącznym, przejęcie wiersza (rekordu) na własność – można czytać, modyfikować, ale inne transakcje nie mają dostępu do tych danych.
      \item \textbf{U-LOCK} - Update (upgrade) lock. Może być tylko jeden na strone (podobnie jak X-LOCK). Transakcja (A) przejmuje dane na wyłączność, ale inne transakcje (B, C, D) mogą zakładać na nie S-LOCKi i odczytywać. \\ Jeśli A chce modyfikować dane, musi podnieść poziom blokady na X-LOCK, ale musi poczekać na zdjęcie przez pozostałe (B, C, D) blokad S-LOCK. U-LOCK daje A tą przewagę, że ona jako pierwsza będzie modyfikować dane, bo nikt inny nie założy X-LOCKa – bo U-LOCK blokuje strone.
      \end{itemize}

  % --- poziomy izolacji
  \section*{Poziomy izolacji}
      \begin{itemize}
          \item \textbf{RO} (\emph{Read Only})
          \begin{itemize}
              \item Nie ma S-LOCKa.
              \item Tworzy się “select snapshot”.
              \item Odczytuje się dane spojne w danej chwil.
              \item Innymi słowy - nie zakłada żadnych blokad, ale nie pozwala na modyfikacje.
              \item Poziom przeznaczony do odczytów, które muszą zawierać dane z tej samej chwili.
              \item Możliwość użycia tego trybu trzeba odblokować przez \emph{SET READONLY ON}
              \item Przekazuje „migawkę" danych z chwili zadania zapytania.
              \item Używa specjalnych plików .HIS do przechowywania migawki.
          \end{itemize}
          \item \textbf{RL} (\emph{Release Lock})
          \begin{itemize}
              \item Wypełnia bufor wejściowy danymi.
              \item Zakłada blokadę na stronę na której jest kursor.
              \item Zwalnia blokadę natychmiast po zakończeniu operacji.
              \item Zapewnia dobry wielodostęp.
              \item Poziom przeznaczony do pokazywania wielu wierszy.
              \item Nie ma S-LOCKa, ale w praktyce DBMS zakłada go na chwilę, odczytuje spójne dane, a następnie zwalnia.
          \end{itemize}
          \item \textbf{CS} (\emph{Cursor Stability})
          \begin{itemize}
              \item Pobiera do bufora tylko pojedynczy wiersz.
              \item Blokowana jest tylko „bieżąca" strona - ta na której jest kursor.
              \item Blokada jest zdejmowana gdy kursor przejdzie na inną stronę.
              \item Dane między kolejnymi odczytami mogą być zmienione przez innych użytkowników.             \item Generuje duży ruch na sieci, ale nie blokuje dostępu innym.
              \item Poziom przeznaczony do zmian pojedynczych wierszy.
              \item Wykorzystywane przy wybieraniu wierszy do modyfikacji ( użytkownik je sobie przegląda w aplikacji i wprowadza zmiany, a poźniej zapisuje ). S-LOCK, U-LOCK, X-LOCK.
          \end{itemize}
          \item \textbf{RR} (\emph{Repetable Read})
          \begin{itemize}
              \item Wypełnia bufor wejściowy danymi.
              \item Wszystkie przeczytane strony zostają zablokowane.
              \item Gwarantuje powtarzalność odczytów wiersza w ramach transakcji.
              \item Jest to domyślny poziom izolacji.
              \item Pełne blokowanie S-LOCK, U-LOCK, X-LOCK.
          \end{itemize}
      \end{itemize}


  % --- Fazy realizacji zapytania SQL
  \part*{Fazy realizacji zapytania SQL}
      \begin{itemize}
      \item \textbf{Faza PREPARE}		% === PREPARE ===
          \begin{itemize}
          \item Przesłanie komendy do serwera
          \item Analiza syntaktyczna
          \item Sprawdzanie istnienia obiektów (tablice, perspektywy, kolumny)
          \item Sprawdzenie uprawnień (poziom uprawnień do operowania na zawartościach tabel)
          \item Podstawienia (np. zastąpienie perspektyw, synonimów)
              \begin{itemize}
              \item Pobranie statystyk z katalogu systemowego
              \item Analiza możliwych ścieżek realizacji wyszukiwania (access path)
              \item Wybór ścieżki o najniższym koszcie wykonania (SQL cost)
              \end{itemize}
          \item Tłumaczenie na ciąg wywołań wykonywalnych modułów
          \end{itemize}
      \item \textbf{Faza EXECUTE}		% === EXECUTE ===
          \begin{itemize}
          \item Wysłanie zmiennych wiążących (bind variables)
          \item Wykonanie komendy (rozpoczęcie tworzenia zbioru \emph{result set})
          \end{itemize}
      \item \textbf{Faza FETCH}		% === FETCH ===
          \begin{itemize}
          \item pobieranie wierszy rozwiązania (każdy “FETCH” to pobranie wiersza ze zbioru \emph{result set})
          \end{itemize}
      \end{itemize}


  % ===================================================================
  % === ZARZĄDZANIE DOSTĘPEM
  % ===================================================================
  \part*{Zarządzanie dostępem}

  % --- Autoryzacja
  \section*{Autoryzacja}
      \begin{itemize}
      \item Użytkownicy są identyfikowani przez nazwy i hasła 
      \item Identyfikator \textbf{PUBLIC} używany jest do przyznania przywilejów dla wszystkich użytkowników
      \end{itemize}

  % --- Poziomy uprawnień i róznice między nimi
  \section*{Poziomy uprawnień i róznice między nimi}
      \begin{itemize}
          \item \textbf{SYSADM}
          \begin{itemize}
              \item Pełna kontrola nad całą bazą.
              \item Potrafi wszytsko, zarządza użytkownikami, hasłami i poziomami uprzywilejowania.
              \item Jest tylko jeden - nie można utworzyć nowych \emph{sysadmów}.
              \item Jako jedyny może tworzyć nowych użytkowników.
          \end{itemize}
          \item \textbf{DBA}
          \begin{itemize}
              \item Pełny dostęp do bazy.
              \item Zarządzanie przywilejami - może przyznawać i odbierać prawa do obiektów.
          \end{itemize}
          \item \textbf{RESOURCE}
          \begin{itemize}
              \item Tworzenie i usuwanie stworzonych przez siebie tablic.
              \item Odbieranie i nadawanie praw do ww. tablic innym użytkownikom.
          \end{itemize}
          \item \textbf{CONNECT}
          \begin{itemize}
              \item Połączenie z bazą danych.
              \item Operacje na tablicach do których ma odpowienie uprawnienia.
              \item Tworzenie synonimów.
              \item Dostęp tylko według przyznanych przywilejów.
          \end{itemize}
      \end{itemize}

  % --- Przywileje
  \section*{Przywileje dostępu}
      \begin{itemize}
          \item Wyszukiwanie (\emph{SELECT}) 
          \item Zmiana danych (\emph{INSERT}, \emph{DELETE}, \emph{UPDATE}), można określić dostęp do kolumn 
          \item Zmiana struktury (\emph{ALTER}) 
          \item Tworzenie i usuwanie indeksów (\emph{INDEX})
          \item Właściciel tablicy ma do niej wszystkie prawa
      \end{itemize}

  % --- Prawa do obiektów bazy danych
  % === DO UZUPEŁNIENIA ==============
  \section*{Prawa do obiektów bazy danych}
      \begin{itemize}
      \item Prawa do obiektów bazy danych (tablica, perspektywa, procedura bazy danych)
      \item Sposób nadawania uprawnień do obiektów (np. przykrywanie poprzez synonimy publiczne i prywatne)
      \item Tablic przez perspektywy dla określonych użytkowników
      \item Możliwość wykonywania procedur z prawami twórcy)
      \end{itemize}

  % --- Zabezpieczenie serwera
  \section*{Zabezpieczenie serwera}
      \begin{itemize}
      \item Komenda \emph{SET SERVER nazwa\_serwera / hasło\_serwera} jest konieczna do wykonania operacji na serwerze:
          \begin{itemize}
          \item Backup i restore 
          \item Tworzenie oraz kasowanie baz i dzienników 
          \item Instalacja oraz deinstalacja bazy danych
          \end{itemize}
      \end{itemize}

  % --- Synonimy i perspektywy
  \section*{Synonimy i perspektywy}
      \begin{itemize}
          \item Dla ułatwienia dostępu do obiektów utworzonych przez innego użytkownika można definiować synonimy poleceniem:
  \center{\emph{CREATE} [\emph{PUBLIC}] \emph{SYNONYM} synonim \emph{FOR} uid.tabela}
          \begin{itemize}
              \item Istnienie publicznego synonimu nie oznacza publicznych praw dostępu! 
              \item Do selektywnego udostępniania tabel można używać perspektyw, w przypadku przywilejów umożliwiających zmiany warto użyć perspektyw \emph{WITH CHECK OPTION}
          \end{itemize}
      \end{itemize}

  % --- Nadawanie i odbieranie przywilejów dostępu
  \section*{Nadawanie i odbieranie przywilejów dostępu}
      \subsection*{Do tablic i perspektyw}
      \begin{itemize}
          \item Przywileje może nadawać i odbierać tylko właściciel, DBA lub SYSADM.
          \item Można odebrać tylko przywileje nadane expilicite
      \end{itemize}


  % ==========================================================================
  % === Sposoby zarządzania dostępem 
  % ==========================================================================
  \part*{Sposoby zarządzania dostępem}

  % --- SQL
  \section*{W SQL}
      \begin{itemize}
          \item Tworzenie użytkowników i nadawanie poziomów dostępu\\\\
          \centerline{\emph{GRANT CONNECT}|\emph{RESOURCE}|\emph{DBA TO} uid \emph{IDENTIFIED BY} passwd;}
          \item Usuwanie użytkowników\\\\ \centerline{\emph{REVOKE CONNECT}|\emph{RESOURCE}|\emph{DBA FROM} uid;}
          \item Przyznawanie przywilejów dostępu do tabeli lub\\\\ \centerline{\emph{GRANT ALL} | lista\_przywilejów \emph{ON} nazwa \emph{TO} uid | \emph{PUBLIC} }
          \item Przywilej \emph{UPDATE} może dotyczyć tylko wybranych kolumn\\\\ \centerline{\emph{UPDATE} ( lista\_kolumn ) }
          \item Odbieranie przywilejów\\\\ \centerline{\emph{REVOKE ALL} | lista\_przywilejów \emph{FROM} uid | \emph{PUBLIC}}
          \item Użytkowników i przywileje można oglądać przez perspektywy:
          \begin{itemize}
              \item \emph{SYSUSERAUTH}
              \item \emph{SYSTABAUTH}
              \item \emph{SYSCOLAUTH}
          \end{itemize}
      \end{itemize}

  % --- Różnice pomiędzy modelem logicznym i fizycznym bazy danych
  % === DO UZUPEŁNIENIA ==============
  \section*{Różnice pomiędzy modelem logicznym i fizycznym bazy danych}
      \begin{itemize}
      \item W modelu fizycznym pracyjemy już na konkretnych tablicach, następuje migracja kluczy obcych, w logicznym występują encje.
      \end{itemize}

  % --- Różnice w dostępie do danych za pośrednictwem indeksów B+tree i cluster hashed w SQLBase
  % === DO UZUPEŁNIENIA ==============
  \section*{Różnice w dostępie do danych za pośrednictwem indeksów B+tree i cluster hashed w SQLBase}
      \begin{itemize}
      \item działanie optymalizatora "wybierającego" metodę dostępu na podstawie postaci predykatu w WHERE
      \item dla predykatu wycinającego ileś wierszy (np. Typu “<” albo between) wybiera się index B+ jesli istnieje, bo jest szybszy (wchodzi się “do liści” i wycina odpowiednie poddrzewo z roziwiązaniami).
      \item Wstawianie jest szybsze dla cluster hashed.asd
      \item 
      \end{itemize}

  % === 

  %===============================================================================	
  %*******************************************************************************
  %===============================================================================
  \end{document}